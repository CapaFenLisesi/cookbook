\section{Cheese Soup}
% Untertitel
\begin{centering}
% Wie viele werden satt?

\end{centering}
%\textbf{Zutaten:}
\begin{table}[H]
\centering
\begin{tabular*}{1\textwidth}{rlrl}
%& && \\
17.6\,oz & ground beef meat & 7\,oz & soft cheese with herbs \\
2 & large onions & 7\,oz & soft cheese with cream \\
2  & cloves garlic & & freshly ground pepper \\
3 sticks & leek &1\,ts& fenugreek \\
2\,l & vegetable broth &17,6\,oz & fresh champignons \\
 & a little oil & & maybe a hint of red wine or cognac\\
\end{tabular*}
\end{table}
%Zubereitung:
\begin{Notes}
\item Cut onions into dices, champignons and leek into slices.
\item Roast ground meat in oil and add the onions. Together with garlic let braise thoroughly.
\item Add leek and let braise as well, then add vegetable broth and champignons. Boil lightly for about 10\,min.
\item Dissolve both kinds of soft cheese in the soup. Season with freshly ground pepper and fenugreek. Optionally add a twist of red wine or cognac.
\end{Notes}
Enjoy!
\vfill
\begin{figure}[H]
  \centering
  \includegraphics[width=0.65\textwidth]{kaesesuppe.jpg}
\end{figure}
\newpage
\section*{Käsesuppe}
% Untertitel
\begin{centering}
% Wie viele werden satt?

\end{centering}
%\textbf{Zutaten:}
\begin{table}[H]
\centering
\begin{tabular*}{1\textwidth}{rlrl}
%& && \\
500\,g & Rindergehacktes &200\,g & Kräuterschmelzkäse \\
2 & große Zwiebeln & 200\,g & Sahneschmelzkäse \\
2 & Knoblauchzehen & & frisch gemahlener Pfeffer \\
3 Stangen & Lauch & 1\,ts &Schabzigerklee (ersatzweise Bockshornklee) \\
2\,l & Gemüsebrühe &500\,g & frische Champignons \\
 & etwas Öl & & evtl. ein Schuß Rotwein oder Cognac\\
\end{tabular*}
\end{table}
%Zubereitung:
\begin{Notes}
\item Die Zwiebeln würfeln, die Champignons und den Lauch in Scheiben schneiden. 
\item Das Hackfleisch in Öl anbraten und die Zwiebeln hinzufügen. Zusammen mit dem Knoblauch durchschmoren.
\item Den Lauch mitschmoren, dann die Gemüsebrühe und die Champignons hinzufügen und alles ca. 10\,min köcheln lassen.
\item Die beiden Schmelkäsesorten in der Suppe auflösen, mit frisch gemahlenem Pfeffer und dem Schabzingerklee würzen. Evtl. mit einem Schuß Rotwein oder Cognac abschmecken.  
\end{Notes}
Guten Appetit!
\vfill
\begin{figure}[H]
  \centering
  \includegraphics[width=0.65\textwidth]{kaesesuppe.jpg}
\end{figure}
\newpage
