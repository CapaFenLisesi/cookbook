% Hier steht der Rezepttitel
\section{K\"{u}rbissuppe}
% Untertitel
\begin{centering}
% Danach die Zutaten in Tabellenform
% Wie viele werden satt?
F\"{u}r ? Hungrige
\end{centering}
%\textbf{Zutaten:}
\begin{table}[H]
  \centering
    % eine Tabelle mit insgesamt 4 Spalten, falls mehr Zutaten benoetigt werden
    % links: Menge, rechts: Zutat
  \begin{tabular*}{1\textwidth}{rlrl}
      %&  &&  \\
    2 & Zwiebeln  &&Paprika, edels\"{u}{\ss} \\
    600\,g & Hokkaido-K\"{u}rbis & 300\,ml & Orangensaft\\
    200\,g & M\"{o}hren & 700\,ml &\ Gem\"{u}sebr\"{u}he \\
    2\,El & \"{O}l &&\\
  \end{tabular*}
\end{table}

%Zubereitung:
\begin{Notes}
\item Hokkaido in 2\,cm gro{\ss}e W\"{u}rfel schneiden, M\"{o}hren in Scheiben.
\item Gem\"{u}se 5\,min in \"{O}l and\"{u}nsten, w\"{u}rzen, Fl\"{u}ssigkeit
  dazugeben und ca. 20\,min k\"{o}cheln lassen.
\item Anschlie{\ss}end p\"{u}rieren.
\end{Notes}




