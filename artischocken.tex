\section{Artichokes}

\begin{centering}

\end{centering}

\begin{table}[H]
  \centering
  \begin{tabular*}{1\textwidth}{rlrl}
      % &  & &  \\
& artichockes & & parmesan \\
 & parsley & & salt, pepper \\
 & garlic & & lemon juice (concentrate) \\
 & vegetable broth powder & $\nicefrac{1}{2}$-1 large glasses & white wine \\
\end{tabular*}
\end{table}

%Zubereitung:
\begin{Notes}
\item Cut off stem of the artichockes, keep $\nicefrac{1}{4}$ of the stem. Tear away the outermost leafs of the artichockes and slacken them (pull leaves apart lightly).
\item Place artichocks in a large pot with water and lemon juice.
\item For the paste: Combine olive oil with chopped parsley, garlic, parmesan, salt and pepper.
\item Take artichocks out of the pot and allow the excess water to drip off. Fill with paste and press in small additional dice of parmesan.
\item Lay artichocks and the saved $\nicefrac{1}{4}$ stems in a large pot and fill up with water until half covered. Add 2-3\,ts vegetable broth powder and boil until the water is reduced to about $\nicefrac{1}{4}$.
\item Now add $\nicefrac{1}{2}$-1 large glasses of white wine and let boil a little more.
  \item The artichockes are done, when the leafes can be pulled off easily and the pulp can easily be pulled off with your teeth.
\end{Notes}
\vfill
\begin{figure}[H]
  \centering
  \includegraphics[width=0.5\textwidth]{artischocken.jpg}
\end{figure}
\newpage

\section*{Artischocken}

\begin{centering}

\end{centering}

\begin{table}[H]
  \centering
    
  \begin{tabular*}{1\textwidth}{rlrl}
      % &  & &  \\
 & Artischocken & & Parmesan \\
 & Petersilie & & Salz, Pfeffer \\
 & Knoblauch & & Zitronensaft(konzentrat) \\
 & Gemüsebrühe & $\nicefrac{1}{2}$-1 große Gläser & Weißwein \\
  \end{tabular*}
\end{table}

%Zubereitung:
\begin{Notes}
\item Stiel der Artischocke abschneiden, $\nicefrac{1}{4}$ des Stiels behalten. Äußerste Blätter der Artischocken abreißen, Artischocken lockern (die Blätter leicht auseinander ziehen).
\item Artischocken in einem großen Topf mit Wasser und Zitronensaft einlegen.
\item Für die Paste: Olivenöl mit Petersilie, Knoblauch, Parmesan, Salz und Pfeffer anrühren.
\item Die Artischocken aus dem Topf nehmen und abtropfen lassen. Artischocken mit der Paste füllen und nachträglich noch kleine Parmesanwürfel hineindrücken.
\item Artischocken und die $\nicefrac{1}{4}$-Stiele in einen großen Topf legen und mit Wasser füllen bis sie zur Hälfte bedeckt sind. 2-3\,TL Gemüsebrühe dazu geben und kochen lassen, bis das Wasser auf ca. $\nicefrac{1}{4}$ reduziert ist.
\item Dann $\nicefrac{1}{2}$-1 große Gläser Weißwein dazu geben und noch etwas kochen lassen.
  \item Fertig sind die Artischocken, wenn man die Blätter leicht abziehen kann und das Fruchtfleisch sich ohne Mühe mit den Zähnen abziehen lässt.
\end{Notes}
\vfill
\begin{figure}[H]
  \centering
  \includegraphics[width=0.5\textwidth]{artischocken.jpg}
\end{figure}
\newpage

