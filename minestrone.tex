\section{Minestrone}

\begin{centering}

  None more often, none tastes better - To fill 4 eaters

\end{centering}

\begin{table}[H]
  \centering

  \begin{tabular*}{1\textwidth}{rlrl}
      % &  & &  \\
    4\nicefrac{1}{2} cups/17.6\,oz & mixed vegetables (best:  celery
    stalks,&&\\
    & zucchini, carrots, young savoy cabbage&&\\
    &spinach)&&\\
    1\nicefrac{1}{2}\,l & vegetable or meat broth & 4 & tomatoes\\
    7\,oz & maccaroni &1 & onion\\
    1 tin/12.7\,oz & boiled white beans & 2 cloves & garlic\\
    & salt, pepper freshly ground & 3\,tbs & olive oil \\
    4\,tbs & freshly grated parmesan\\
    &or pecorino &&\\
  \end{tabular*}
\end{table}

%Zubereitung:
\begin{Notes}
\item Clean all vegetables, peel carrots, slice vegetables in dices (celery, zucchini, carrots and tomatoes) or stripes (savoy cabbage, spinach). Peel onion and garlic and cut very fine.
\item Put a large stockpot on the stove. Let 1\,tbs oil get warm in it. Stir in onions and garlic and saut\'{e}. Add vegetable broth, mix in vegetables and let all become nice and hot.
\item The veggies has to boil lightly now for about 15\,min. Therefore turn down to medium heat. Put on the lid omly half (clamp the spoon so the lid doesn't slip).
\item Break pasta in pieces (about 5\,cm long). Rinse beans with cold water in a sieve, for the fluid is murky and slimy. Add pasta and beans to soup and let boil for about 10 more minutes, until the pasta is al dente.
\item Try the minestrone, is salt and pepper missing? Ladle soup into soup plates, sprinkle with the left oil and grated cheese and ready is the most italian soup of all times.
\end{Notes}
How much time to take: 45\,min, of which 30 are work
Goes well with: fresh white bread, toasted if you like. Also nice: A small spoon of pesto for every plate.

\begin{figure}[H]
  \centering
  \reflectbox{\includegraphics[width=0.4\textwidth]{Zucchini-Whole.jpg}}
\end{figure}
\newpage

\section*{Minestrone}

\begin{centering}

  Keine gibt's \"{o}fter, keine schmeckt besser - F\"{u}r 4 zum Sattessen

\end{centering}

\begin{table}[H]
  \centering

  \begin{tabular*}{1\textwidth}{rlrl}
      % &  & &  \\
    500\,g & gemischtes Gem\"{u}se (am besten: Stangensellerie,& &\\
    & Zucchini, M\"{o}hren, junger Wirsing,& & \\
    & Spinat)& & \\
    1\nicefrac{1}{2}\,l & Gem\"{u}se- oder Fleischbr\"{u}he & 4 & Tomaten \\
    200\,g & Maccaroni & 1 & Zwiebel\\
    1 Dose & gegarte wei{\ss}e Bohnen (360\,g) & 2 & Knoblauchzehen \\ & Salz,
    Pfeffer aus der M\"{u}hle & 3\,EL & Oliven\"{o}l\\
    4\,EL & frisch geriebener Parmesan oder Pecorino &&\\
  \end{tabular*}
\end{table}

%Zubereitung:
\begin{Notes}
\item Alles Gem\"{u}se waschen, M\"{o}hren sch\"{a}len, Gem\"{u}se in W\"{u}rfel (Sellerie, Zucchini, M\"{o}hren und Tomaten) oder Streifen (Wirsing, Spinat) schneiden. Zwiebeln und Knoblauch sch\"{a}len und ganz fein schneiden.
\item Einen gro{\ss}en Topf auf den Herd stellen. 1\,EL \"{o}l darin warm werden lassen. Zwiebeln und Knoblauch reinr\"{u}hren und anbraten. Gem\"{u}sebr\"{u}he dazugie{\ss}en, Gem\"{u}se untermischen und alles sch\"{o}n hei{\ss} werden lassen.
\item Das Gem\"{u}se soll jetzt ungef\"{a}hr 15\,min k\"{o}cheln. Also die Hitze auf mittlere Stufe stellen. Und den Deckel nur halb auflegen (Kochl\"{o}ffel dazwischen klemmen, dann rutscht der Deckel nicht).
\item Nudeln in St\"{u}cke brechen (ungef\"{a}hr 5\,cm lang). Bohnen im Sieb kalt absp\"{u}len, die Garfl\"{u}ssigkeit ist n\"{a}mlich tr\"{u}b und glibbrig. Nudeln und Bohnen zur Suppe geben und nochmal ungef\"{a}hr 10\,min weiterk\"{o}cheln lassen, bis die Nudeln al dente sind.
\item Minestrone probieren, fehlt noch Salz und Pfeffer? Suppe in tiefe Teller sch\"{o}pfen, das \"{u}brige \"{o}l dar\"{u}berlaufen lassen, K\"{a}se drauf streuen und fertig ist die italienischste Suppe aller Zeiten.
\end{Notes}

Soviel Zeit muss sein: 45\,min, davon 30 zu tun
Das schmeckt dazu: frisches Wei{\ss}brot, wer mag, kann es leicht anr\"{o}sten. Auch gut dazu: pro Teller ein L\"{o}ffelchen Pesto
\newpage
