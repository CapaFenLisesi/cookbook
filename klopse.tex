% Hier steht der Rezepttitel
\section{Omas K\"{o}nigsberger Klopse}
% Untertitel
\begin{centering}
% Danach die Zutaten in Tabellenform
% Wie viele werden satt?
F\"{u}r 10 Hungrige
%\textbf{Zutaten:}
\begin{table}[H]
  \centering
    % eine Tabelle mit insgesamt 4 Spalten, falls mehr Zutaten benoetigt werden
    % links: Menge, rechts: Zutat
  \begin{tabular*}{1\textwidth}{rlrl}
      %&  &&  \\
    1\,kg & Hackflesich  &&Zitronensaft \\
    1\nicefrac{1}{2} & Br\"{o}tchen & 2 Becher & Saure Sahne\\
    2 Becher & Cr\`{e}me fra\^{i}che &100-150\,g&Kapern\\
    &Milch zum Einweichen &2x &helle So{\ss}e (Fertigprodukt)\\
    2-3 & Zwiebeln & &Butter\\
    2 & Eier & 1\,$\ell$ & Gem\"{u}sebr\"{u}he\\
  \end{tabular*}
\end{table}
\end{centering}
%Zubereitung:
\begin{Notes}
\item Am Vorabend Br\"{o}tchen in Milch einweichen, kr\"{a}ftig ausdr\"{u}cken.
  \item Zwiebeln w\"{u}rfeln und glasig anbraten, Hackfleisch dazugeben. Mit
    Pfeffer, Salz, 2 Eiern, und Br\"{o}tchen durchmengen.
  \item Klopse aus der Masse formen und in Gem\"{u}sebr\"{u}he 20\,min ziehen
    lassen.
  \item Klopse entfernen und die helle So{\ss}e in der Br\"{u}he anr\"{u}hren.
    Saure Sahne und Cr\`{e}me fra\^{i}che dazugeben und mit den Klopsen,
    zusammen mit den Kapern, aufkochen.
\end{Notes}

Lorem Ipsum dolor sit amet\ldots




