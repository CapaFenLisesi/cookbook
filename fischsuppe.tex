\section{Fish Soup}
% Untertitel
\begin{centering}
% Danach die Zutaten in Tabellenform
% Wie viele werden satt?
For 3 hungry ones
\end{centering}
%\textbf{Zutaten:}
\begin{table}[H]
\centering
% eine Tabelle mit insgesamt 4 Spalten, falls mehr Zutaten benoetigt werden
% links: Menge, rechts: Zutat
\begin{tabular*}{1\textwidth}{rlrl}
%& && \\
3 & onions & 1.7\,cups/14\,oz & chopped tomatoes \\
2\,tbs & olive oil &3.5\,oz & zander filet\\
1 & zucchini, medium sized & 7\,oz & redfish filett\\
3 & tomatoes & 8.8\,oz & catfish filet\\
0.8 cups/6.8\,oz & fish stock && \\
&For garlic mayonaise:&&\\
& mayonaise to taste & 1 & clove of garlic\\
\end{tabular*}
\end{table}
%Zubereitung:
Lorem Ipsum dolor sit amet\ldots


% Hier steht der Rezepttitel
\section*{Fischsuppe}
% Untertitel
\begin{centering}
% Danach die Zutaten in Tabellenform
% Wie viele werden satt?
F\"{u}r 3 Hungrige
\end{centering}
%\textbf{Zutaten:}
\begin{table}[H]
\centering
% eine Tabelle mit insgesamt 4 Spalten, falls mehr Zutaten benoetigt werden
% links: Menge, rechts: Zutat
\begin{tabular*}{1\textwidth}{rlrl}
%& && \\
3 & Zwiebeln & 1\,Dose & gehackte Tomaten\\
2\,EL & Olivenöl &100\,g & Zanderfilet\\
1 & Zucchini, mittelgroß & 200\,g & Rotbarschfilet\\
3 & Tomaten & 250\,g & Pangasiusfilet\\
200\,ml & Fischfond && \\
&Für die Knoblauchmayonaise:&&\\
& Mayonaise nach Geschmack & 1 & Knoblauchzehe\\
\end{tabular*}
\end{table}
%Zubereitung:
Lorem Ipsum dolor sit amet\ldots

