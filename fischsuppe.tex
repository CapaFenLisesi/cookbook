\section{Fish Soup}
% Untertitel
\begin{centering}
% Danach die Zutaten in Tabellenform
% Wie viele werden satt?
For 3 hungry ones
\end{centering}
%\textbf{Zutaten:}
\begin{table}[H]
\centering
% eine Tabelle mit insgesamt 4 Spalten, falls mehr Zutaten benoetigt werden
% links: Menge, rechts: Zutat
\begin{tabular*}{1\textwidth}{rlrl}
%& && \\
3 & onions & 1.7\,cups/14\,oz & chopped tomatoes \\
2\,tbs & olive oil &3.5\,oz & zander filet\\
1 & zucchini, medium sized & 7\,oz & redfish filet\\
3 & tomatoes & 8.8\,oz & catfish filet\\
0.8 cups/6.8\,oz & fish stock && \\
&For garlic mayonaise:&&\\
& mayonaise to taste & 1 & clove of garlic\\
\end{tabular*}
\end{table}
%Zubereitung:
\begin{Notes}
\item Finely dice the onion and cut the vegetables in larger dice. Cut the fish filet in 1\,in
  pieces. 
\item Heat the olive oil in a sufficeintly large cooking pot. When hot saut\'{e}
  the onion and add the vegetables.
\item Deglaze the contents with fish fond and the chopped tomatoes and let cook
  for about 10\,min. 
\item In between chop the garlic very finely and mix with mayonaise
\item Add the fish to the rest of the soup and let steep until done. Season to taste with eg.
  rosemary or thyme. Salt and pepper and serve together with mayonaise and
  sliced baguette.
\end{Notes}
\vfill
\begin{figure}[H]
  \centering
  \reflectbox{\includegraphics[width=0.5\textwidth]{fish.jpg}}
\end{figure}
\newpage

% Hier steht der Rezepttitel
\section*{Fischsuppe}
% Untertitel
\begin{centering}
% Danach die Zutaten in Tabellenform
% Wie viele werden satt?
F\"{u}r 3 Hungrige
\end{centering}
%\textbf{Zutaten:}
\begin{table}[H]
\centering
% eine Tabelle mit insgesamt 4 Spalten, falls mehr Zutaten benoetigt werden
% links: Menge, rechts: Zutat
\begin{tabular*}{1\textwidth}{rlrl}
%& && \\
3 & Zwiebeln & 1\,Dose & gehackte Tomaten\\
2\,EL & Olivenöl &100\,g & Zanderfilet\\
1 & Zucchini, mittelgroß & 200\,g & Rotbarschfilet\\
3 & Tomaten & 250\,g & Pangasiusfilet\\
200\,ml & Fischfond && \\
&Für die Knoblauchmayonaise:&&\\
& Mayonaise nach Geschmack & 1 & Knoblauchzehe\\
\end{tabular*}
\end{table}
%Zubereitung:
\begin{Notes}
\item Die Zwiebel fein w\"{u}rfeln, das Gem\"{u}se etwas gr\"{o}ber. Den Fisch
  in ca. 2.5\,cm gro{\ss}e W\"{u}rfel schneiden. 
\item In einem ausreichend gro{\ss}en Topf das Oliven\"{o}l erhitzen und die
  Zwiebelw\"{u}rfel bei mittlerer Hitze anschwitzen. Sind die Zwiebeln glasig,
  das restliche Gem\"{u}se dazugeben und ebenfalls anbraten. 
\item Den Topfinhalt mit Fischfond und gehackten Tomaten abl\"{o}schen und
  ungef\"{a}hr 10 \,min kochen lassen. 
\item In der Zwischenzeit kann der Knoblauch sehr fein gehackt und mit der
  Mayonaise gemischt werden. 
\item Sind die 10\,min um, das Fischfilet hinzuf\"{u}gen und gar ziehen lassen.
  Nach Belieben mit z.B. Rosmarin oder Thymian w\"{u}rzen, salzen und pfeffern
  nicht vergessen!
  Zusammen mit der Mayonaise und Baguettescheiben servieren.
\end{Notes}
\vfill
\begin{figure}[H]
  \centering
  \includegraphics[width=0.5\textwidth]{fish.jpg}
\end{figure}
\newpage
